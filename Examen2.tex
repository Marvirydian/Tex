\documentclass[12pt,a4paper]{article}
\usepackage[latin1]{inputenc}
\usepackage[spanish]{babel}
\usepackage{amsmath}
\usepackage{amsfonts}
\usepackage{amssymb}
\usepackage{makeidx}
\usepackage{graphicx}
\usepackage{fourier}
\usepackage[left=2cm,right=2cm,top=2cm,bottom=2cm]{geometry}
\begin{document}
\title{Recuperaci�n. Examen 2}
\author{Castro Luna Diana Vanessa.}
\date{}
\maketitle
1. Una muestra de gas se expande desde una presi�n inicial y volumen de 10Pa y 1 $m^{3}$ a un volumen final de 2$m^{3}$. Durante la expansi�n, la presi�n y el volumen est�n relacionados por la ecuaci�n $p=aV^{2}$, donde $a=10N/m^{8}$. Determine el trabajo hecho por el gas durante la expansi�n.\
\\
\\
SOLUCI�N
\\
\\
2.Una pieza de metal de 18kg inicialmente a temperatura de 180�C se sumerge en agua en un contenedor hecho del mismo material. El contenedor tiene una masa de 3.6 kg y contiene 14kg de agua. El contenedor y el agua inicialmente tienen una temperatura de 16�C y la temperatura final del sistema (aislado) es de 18�C. �Cu�l es el calor espec�fico $c_{m}$ del metal? El calor espec�fico del agua es $4,178x10^{3} J/kg K$.
\\
\\
SOLUCI�N
\\
Usando que el calor cedido es igual al calor absorbido, tenemos lo siguiente.\\
La pieza de metal va a ceder al calor al agua y recipiente, entonces:
\[Q_{1}= m_{m} C_{em} (T_{f}-T_{i})\]
\[Q_{1}=(18kg)C_{em} (180-18)\circ C\]
$\Rightarrow$
\[Q_{1}=18(162)C_{em}=2916 C_{em}\]
el agua en el contenedor y contenedor absorben calor, as� que:
\[Q_{2}= m_{a} C_{ea} (T_{f}-T_{i})\]
\[Q_{2}= (14kg)(4.178x10^{3}J/kg*K)(18-16)\]
\[Q_{2}=116,844\]
y
\[Q_{3}= m_{c} C_{em} (T_{f}-T_{i})\]
\[Q_{3}=(3.6kg)(2\circ C)C_{em}\]
\[Q_{3}=/7.2 C_{em}\]
Entonces el calor espec�fico del metal es:
\[Q_{1}= Q_{2}+Q_{3}\]
\[2916C_{em}=116,844+7.2C_{em}\]
\[(2916-7.2)C_{em}=116,844\]
\[C_{em}=40.16J/kgK\]
\\
\\
3.Un cilindro contiene 0.250 moles de di�xido de carbono $CO_{2}$ gaseoso a  una temperatura de 27�C. El cilindro cuenta con un pist�n sin fricci�n, el cual mantiene una presi�n constante sobre el gas de $1 atm$. El gas se calienta hasta que su temperatura aumenta a 127�C. Suponga que el $CO_{2}$ se puede tratar como un gas ideal ($C_{v}=28.46 J/mol*K$). A) Dibuje una gr�fica p-V. B)�Cu�nto trabajo efect�a el gas en este proceso? C)�Sobre que se efect�a ese trabajo? D)�Cu�nto cambia la energ�a interna del gas? E)�C�anto calor se suministro al gas? F)�C�anto trabajo se hubiera efectuada si la presi�n hubiera sido $0.5 atm$?
\\
\\
SOLUCI�N
\\
A) Gr�fica de P-V\\
\includegraphics[scale=.8]{graf.jpg} 
\\
\\
B) El trabajo est� dado por:
\[W=pV_{2}-pV_{1}\]
\[W=nRT_{2}-nRT_{1}\]
\[W=(0.250mol)(8.3145J/molK)(100K)\]
$\Rightarrow$
\[W=207.86J\]
\\
\\
C) El trabajo es el que el gas ejerce sobre el pist�n, adem�s este es positivo.
\\
\\
D) Tenemos que el cambio en la energ�a ser�:
\[\Delta U = nC_{v} \Delta T\]
$\Rightarrow$
\[\Delta U = (0.250mol)(28.46 J/molK)(100K)\]
\[\Delta U=711.5 J\]
\\
\\
D) Sabemos que $\Delta U = Q-W$ entonces el calor suministrado es:
\[Q=\Delta U + W\]
\[Q=711.5 J + 207.8 J\]
$\Rightarrow$
\[Q=919.3 J\]
\\
\\
F) Como vimos en el inciso A) el trabajo no depende de la presi�n, por lo tanto el trabajo ser�a el mismo  $W=207.86J$.
\end{document}
